\section{Conclusiones}

Este proyecto ha demostrado la viabilidad de integrar un sistema generativo de imágenes a partir de descripciones textuales dentro de una plataforma multimedia existente. A lo largo del desarrollo, se ha abordado con éxito la construcción de un servicio basado en modelos de difusión preentrenados, encapsulado en una API REST eficiente y gestionado mediante herramientas modernas como FastAPI, Poetry y Docker. Esta integración ha permitido establecer un puente entre el entorno Python de generación y la interfaz Java del sistema JMR, facilitando una experiencia fluida para el usuario final.

Uno de los aspectos más enriquecedores del proyecto ha sido, sin duda, la fase de experimentación con distintas arquitecturas generativas. Desde los primeros intentos con GANs y cGANs hasta la posterior exploración de AttnGAN y, finalmente, el uso de modelos basados en difusión como Stable Diffusion, cada etapa ha aportado un aprendizaje profundo sobre los retos técnicos, las limitaciones de cada enfoque y las posibilidades que ofrecen. Este proceso experimental ha sido esencial no solo para llegar a una solución funcional, sino también para consolidar una comprensión crítica del funcionamiento interno de los modelos generativos y sus implicaciones prácticas en la generación de contenido visual.

El sistema desarrollado ha cumplido satisfactoriamente con los objetivos planteados, permitiendo generar imágenes coherentes a partir de descripciones en lenguaje natural. Además, ha mostrado un comportamiento robusto durante las pruebas, tanto en términos de calidad de las imágenes como en su integración con la plataforma cliente. La validación visual de los resultados, junto con la implementación de un flujo controlado de entrada, generación y visualización, refuerzan la utilidad práctica del sistema como herramienta de consulta visual avanzada.

Durante el desarrollo, también se han aplicado buenas prácticas de diseño, como la separación de responsabilidades entre módulos, la validación estructural de los modelos subidos y el uso de entornos reproducibles. Estas decisiones han contribuido a dotar al sistema de una arquitectura modular y fácilmente extensible, preparada para futuras iteraciones o despliegues en producción.

\subsection{Líneas futuras de trabajo}

Aunque el proyecto alcanza un grado alto de completitud, existen múltiples caminos para su evolución. En primer lugar, se plantea la optimización del rendimiento mediante técnicas de compresión o el uso de modelos más ligeros, con el fin de facilitar su ejecución en dispositivos con recursos limitados. Asimismo, resulta clave incorporar mecanismos de seguridad adicionales, como límites de uso, autenticación de usuarios y validación semántica de los prompts, especialmente en escenarios donde el servicio pueda ser expuesto públicamente.

Otra línea de trabajo prometedora es el entrenamiento personalizado de los modelos con conjuntos de datos específicos, lo que permitiría adaptar la generación de imágenes a dominios concretos, como patrimonio cultural, medicina o diseño de producto. Además, la posibilidad de integrar funciones de edición interactiva o regeneración parcial de imágenes abriría nuevas puertas a la creatividad y el control por parte del usuario.

Finalmente, avanzar hacia la evaluación automática de resultados mediante métricas objetivas y explorar nuevos entornos de despliegue, como arquitecturas serverless o plataformas móviles, supondría un paso importante hacia la consolidación del sistema como herramienta accesible, escalable y útil en contextos reales.