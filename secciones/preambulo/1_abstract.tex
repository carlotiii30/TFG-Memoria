\centering
\textbf{Development of an image retrieval system based on Generative A.I.}
\vspace{0.5cm}

Carlota de la Vega Soriano
\vspace{0.5cm}
\justify
\textbf{Keywords}: image retrieval, generative artificial intelligence, CBIR, GAN, natural language processing
\vspace{1cm}

\justify
\textbf{Abstract}

The increasing volume of multimedia content has made visual information retrieval systems (CBIR) more necessary than ever. Traditionally, these systems have relied on low-level descriptors extracted directly from images, which limits their ability to reflect user intent semantically. To overcome this limitation, this project proposes the integration of generative artificial intelligence techniques as an innovative solution, specifically for generating query images from text descriptions with high semantic content.

The main objective of this work is to develop modules for the Java Multimedia Retrieval (JMR) platform to allow textual queries as input to CBIR systems. To this end, three specific goals were defined: reviewing the state of the art in text-to-image generation, developing algorithms to transform linguistic descriptions into visual representations, and implementing a functional retrieval prototype based on text.

The methodology followed was based on the waterfall model, structured into sequential phases of analysis, design, implementation, and testing. A comparative evaluation of different generative architectures was conducted, including GAN, cGAN, and AttnGAN, along with the use of pretrained models such as Stable Diffusion. For training and evaluation, various datasets were used — MNIST, CIFAR, COCO, and Stanford Dogs — adjusting models to different levels of semantic complexity.

Among the main results, the project validates a functional system capable of transforming text into query images and integrating it into the JMR architecture. The tests show an improvement in users’ ability to formulate expressive queries, making CBIR systems more intuitive and semantically rich.

In conclusion, generative artificial intelligence proves to significantly enhance interaction with visual retrieval systems by enabling more expressive, human-like queries. This approach opens up new possibilities in applications where semantic accuracy in image search is critical, such as education, medicine, or design.