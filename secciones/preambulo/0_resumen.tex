\centering
\textbf{Desarrollo de un sistema de recuperación de imágenes basada en I.A. Generativa}
\vspace{0.5cm}

Carlota de la Vega Soriano
\vspace{0.5cm}
\justify
\textbf{Palabras clave}: recuperación de imágenes, inteligencia artificial generativa, CBIR, GAN, procesamiento de lenguaje natural
\vspace{1cm} 

\justify   
\textbf{Resumen}

El creciente volumen de contenidos multimedia ha hecho cada vez más necesarios los sistemas de recuperación de información visual (CBIR). Tradicionalmente, estos sistemas se han basado en descriptores de bajo nivel extraídos directamente de las imágenes, lo que dificulta que las consultas reflejen de forma semántica las necesidades del usuario. Para superar esta limitación, el presente proyecto plantea la incorporación de técnicas de inteligencia artificial generativa como solución innovadora, concretamente para la generación de imágenes consulta a partir de descripciones textuales con alto contenido semántico.

Este trabajo tiene como objetivo principal el desarrollo de módulos para la plataforma Java Multimedia Retrieval (JMR) que permitan la integración de consultas textuales como mecanismo de entrada en sistemas CBIR. Para ello, se han definido tres objetivos específicos: la revisión del estado del arte en generación de imágenes a partir de texto, el desarrollo de algoritmos que traduzcan descripciones lingüísticas en representaciones visuales, y la implementación de un prototipo funcional de recuperación basado en texto.

La metodología seguida ha estado basada en el modelo en cascada, estructurada en fases secuenciales de análisis, diseño, implementación y pruebas. Se ha realizado una evaluación comparativa de distintas arquitecturas generativas, incluyendo GAN, cGAN y AttnGAN, así como el uso de modelos preentrenados como Stable Diffusion. Para el entrenamiento y evaluación, se han empleado distintos conjuntos de datos, como MNIST, CIFAR, COCO y Stanford Dogs, ajustando los modelos a diferentes niveles de complejidad semántica.

Entre los principales resultados alcanzados, destaca la validación de un sistema funcional capaz de transformar texto en imágenes consulta, y su integración en la arquitectura de la JMR. Las pruebas realizadas evidencian una mejora en la capacidad de formulación de consultas por parte del usuario, acercando el sistema CBIR a una experiencia más intuitiva y semánticamente rica.

Como conclusión, se demuestra que la inteligencia artificial generativa puede enriquecer significativamente la interacción con sistemas de recuperación visual, permitiendo consultas más expresivas y adaptadas al lenguaje humano. Este enfoque abre nuevas posibilidades en aplicaciones donde la precisión semántica en la búsqueda de imágenes es esencial, como la educación, la medicina o el diseño.
