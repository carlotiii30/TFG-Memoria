\section{Diseño}

El diseño del sistema desarrollado ha sido clave para garantizar la viabilidad técnica, la modularidad de los componentes y una experiencia de usuario intuitiva. En este apartado se describen los principales aspectos del diseño arquitectónico, conceptual y de interfaz de la solución implementada, incluyendo tanto el funcionamiento interno de los módulos como su interacción con el usuario final.

\subsection{Diseño de la arquitectura}

El sistema se integra en la plataforma Java Multimedia Retrieval (JMR), ampliando sus capacidades de búsqueda visual mediante la incorporación de un módulo generativo desarrollado en Python. La arquitectura sigue un enfoque modular y desacoplado, en el que los componentes desarrollados en Python se comunican con JMR a través de una API REST.

\begin{itemize}
    \item \textbf{Interfaz de usuario (JMR):} permite introducir descripciones textuales como entrada para la generación de consultas visuales.
    
    \item \textbf{Módulo generativo (Python):} expone una API REST que recibe descripciones textuales y devuelve imágenes generadas mediante modelos de difusión. Este módulo está encapsulado en una aplicación ligera desplegable de forma independiente.
    
    \item \textbf{Módulo de integración (Java):} dentro de JMR, se encarga de enviar peticiones HTTP a la API generativa, recuperar la imagen resultante y tratarla como una consulta visual.
    
    \item \textbf{Módulo CBIR (JMR):} compara la imagen generada con una base de datos de imágenes mediante descriptores visuales y métricas de similitud.
\end{itemize}

\begin{figure}[H]
    \centering
    \includegraphics[width=0.7\textwidth]{nueva_arquitectura}
    \caption{Arquitectura lógica del sistema integrado}
    \label{fig:arquitectura-final}
\end{figure}

Esta arquitectura distribuida permite mejorar la mantenibilidad, facilitar la evolución del modelo generativo y adaptar el sistema a diferentes entornos sin afectar a la aplicación principal.

\subsection{Modelo conceptual}

El modelo conceptual del sistema refleja los elementos fundamentales que intervienen en el proceso de generación y búsqueda visual:

\begin{itemize}
    \item \textbf{Usuario:} introduce una descripción textual a través de la interfaz JMR.
    \item \textbf{Prompt o descripción textual:} entrada en lenguaje natural que sirve como semilla para la generación de una imagen.
    \item \textbf{Imagen generada:} salida del modelo de IA a partir del prompt introducido por el usuario.
    \item \textbf{Resultado de búsqueda:} conjunto de imágenes similares recuperadas por el motor CBIR.
    \item \textbf{Configuración del modelo:} conjunto de parámetros que determinan el comportamiento del generador (modelo base, pasos, guidance, etc.).
\end{itemize}

Este modelo permitió definir los flujos de información, el diseño de las peticiones API y las entidades clave del sistema.

\subsection{Diseño del modelo de clases}

Dado que el sistema combina componentes en dos lenguajes distintos, se ha documentado por separado el modelo de clases tanto del cliente Java como del servidor Python.

\subsubsection{Modelo de clases en Java}

El cliente JMR incluye una serie de clases que gestionan la visualización, el flujo de entrada/salida y la integración con el servidor generativo. El siguiente diagrama resume las principales clases y sus relaciones:

\begin{figure}[H]
    \centering
    \includegraphics[width=0.85\textwidth]{diagrama_clases_java}
    \caption{Diagrama de clases del módulo de integración en Java}
    \label{fig:clases-java}
\end{figure}

\subsubsection{Modelo de clases en Python}

Aunque muchos módulos Python siguen una estructura funcional, se han representado como clases sintéticas para reflejar su cohesión interna y responsabilidades. Además, el script de entrenamiento se encapsula conceptualmente como una clase \texttt{DogTrainer}, facilitando su comprensión y documentación:

\begin{figure}[H]
    \centering
    \includegraphics[width=0.85\textwidth]{diagrama_clases_python}
    \caption{Diagrama de clases del sistema generativo en Python}
    \label{fig:clases-python}
\end{figure}

Estas representaciones permiten comprender de forma estructurada la distribución de responsabilidades y la lógica interna de cada componente, sirviendo de soporte a la implementación modular del sistema.


\subsection{Diseño de la interfaz}
El diseño de la interfaz de usuario ha sido un componente clave para facilitar la interacción con el sistema de generación y búsqueda de imágenes. Este apartado presenta la evolución del diseño, desde los primeros esquemas conceptuales hasta el prototipo final interactivo. Se parte del análisis de flujo de interacción, se muestran los bocetos iniciales y wireframes funcionales, y finalmente se presenta el prototipo desarrollado en Figma. Además, se analizan los principios de usabilidad aplicados para garantizar una experiencia fluida e intuitiva para el usuario final.

\subsubsection{Diagrama de flujo de interacción}
El siguiente diagrama de flujo describe la secuencia de operaciones que se ejecutan desde la introducción del prompt por parte del usuario hasta la obtención de los resultados visuales. Este flujo ilustra la lógica general del sistema, destacando las decisiones clave y la coordinación entre los distintos módulos (interfaz gráfica, API de generación y sistema CBIR).

\begin{figure}[H]
    \centering
    \includegraphics[width=0.6\textwidth]{diagramas/diagrama_flujo.png}
    \caption{Flujo de interacción entre el usuario, la API generativa y el sistema CBIR}
    \label{fig:flujo-interaccion}
\end{figure}

\subsubsection{Bocetos}

Los primeros bocetos se realizaron a mano con el objetivo de definir la estructura inicial de la interfaz, priorizando la disposición de los componentes principales: el área de entrada del prompt, el botón de generación, la galería de resultados y las opciones adicionales de filtrado o guardado. Estos bocetos permitieron una exploración rápida de ideas antes de pasar a herramientas digitales.

\begin{figure}[H]
    \centering
    \includegraphics[width=0.8\textwidth]{bocetos/boceto1.png}
    \caption{Boceto: propuesta inicial del layout general}
    \label{fig:boceto1}
\end{figure}

\begin{figure}[H]
    \centering
    \includegraphics[width=0.8\textwidth]{bocetos/boceto2.png}
    \caption{Boceto: navegación entre secciones}
    \label{fig:boceto2}
\end{figure}

\begin{figure}[H]
    \centering
    \includegraphics[width=0.8\textwidth]{bocetos/boceto3.png}
    \caption{Boceto: detalle de interacción en la vista de resultados}
    \label{fig:boceto3}
\end{figure}

\subsubsection{Wireframes}

A partir de los bocetos iniciales, se desarrollaron wireframes de baja fidelidad utilizando herramientas digitales. Estos permitieron refinar la experiencia de usuario y definir la jerarquía visual de cada componente. Los wireframes presentan la organización funcional de las pantallas clave: generación de imágenes, galería de resultados y acciones disponibles sobre cada imagen.

\begin{figure}[H]
    \centering
    \includegraphics[width=0.8\textwidth]{wireframes/wireframe1.png}
    \caption{Wireframe: pantalla principal con área de generación}
    \label{fig:wireframe1}
\end{figure}

\begin{figure}[H]
    \centering
    \includegraphics[width=0.8\textwidth]{wireframes/wireframe2.png}
    \caption{Wireframe: visualización detallada de una imagen generada}
    \label{fig:wireframe3}
\end{figure}

\begin{figure}[H]
    \centering
    \includegraphics[width=0.8\textwidth]{wireframes/wireframe3.png}
    \caption{Wireframe: pantalla de resultados con opciones de filtrado}
    \label{fig:wireframe2}
\end{figure}

\subsubsection{Prototipo en Figma}

Como parte final del diseño, se elaboró un prototipo funcional en Figma basado en los wireframes anteriores. Este prototipo interactivo permitió simular la navegación entre pantallas, validar la disposición de los elementos y detectar posibles fricciones en la experiencia de usuario. Además, facilitó la comunicación visual con el equipo de desarrollo y la obtención de feedback por parte de usuarios potenciales.

\begin{figure}[H]
    \centering
    \includegraphics[width=0.8\textwidth]{wireframes/wireframe1.png}
    \caption{Captura del prototipo desarrollado en Figma}
    \label{fig:figma-prototype}
\end{figure}

\subsubsection{Usabilidad}

El diseño de la interfaz se ha guiado por principios fundamentales de usabilidad con el objetivo de ofrecer una experiencia accesible, eficiente y satisfactoria para todo tipo de usuarios, independientemente de su nivel técnico. A partir de las historias de usuario definidas (HU1 y HU2), se identificaron tres necesidades clave: introducir descripciones textuales de forma clara, visualizar resultados generados de forma comprensible, y reutilizar esas imágenes para buscar contenido visual similar en la base de datos.

Con base en estos objetivos, se aplicaron los siguientes principios de diseño centrado en el usuario:

\begin{itemize}
    \item \textbf{Interacción simple y directa:} La interfaz principal presenta solo los elementos esenciales para la tarea principal: un campo para introducir el prompt, un botón para iniciar la generación y un área de visualización de la imagen resultante. Esta simplicidad evita la sobrecarga cognitiva y facilita que el usuario comprenda rápidamente cómo usar el sistema sin necesidad de asistencia externa.

    \item \textbf{Retroalimentación inmediata y comprensible:} Tras introducir el prompt, el sistema proporciona indicadores visuales (como animaciones de carga) que confirman que la solicitud está siendo procesada. Al completarse la generación, la imagen se muestra de forma destacada, acompañada de una acción clara para continuar con el flujo (por ejemplo, usarla como consulta en el sistema CBIR). Esto mantiene al usuario informado en todo momento y refuerza su confianza en el sistema.

    \item \textbf{Consistencia visual y funcional:} Se respetaron convenciones de diseño ampliamente reconocidas: uso de etiquetas descriptivas, alineaciones verticales, botones claramente diferenciados y jerarquía visual basada en el tamaño y el color. Esto reduce el tiempo de aprendizaje y evita comportamientos inesperados.

    \item \textbf{Accesibilidad y legibilidad:} Se priorizó el uso de tipografía clara, buen contraste entre texto y fondo, y tamaños adecuados de los elementos interactivos para facilitar el uso tanto en pantallas grandes como en dispositivos de menor tamaño. Asimismo, se evitó el uso de términos técnicos complejos, apostando por un lenguaje neutro y comprensible.

    \item \textbf{Minimización de errores y puntos de fricción:} El flujo está diseñado para prevenir errores comunes, como el envío de prompts vacíos o duplicados. Además, en caso de fallo (por ejemplo, si no se genera una imagen), el sistema informa con mensajes explicativos que permiten al usuario comprender lo ocurrido y actuar en consecuencia.
\end{itemize}

En conjunto, estas decisiones de diseño han permitido construir una interfaz que no solo es funcional, sino también intuitiva y centrada en las necesidades reales del usuario. Esto resulta especialmente importante en un sistema como este, donde se combinan tecnologías avanzadas (IA generativa y recuperación de imágenes) con una interacción aparentemente simple basada en lenguaje natural.
