\subsection{Objetivos}

El presente Trabajo Fin de Grado tiene como objetivo principal el desarrollo de un sistema de recuperación de imágenes (CBIR) que permita realizar consultas a través de descripciones textuales con alto contenido semántico, generando dinámicamente imágenes de referencia mediante técnicas de inteligencia artificial generativa. Este enfoque persigue reducir la brecha entre la intención del usuario y los métodos tradicionales de recuperación, proporcionando una experiencia de búsqueda más natural, intuitiva y flexible.

La investigación y el desarrollo en este ámbito se basan en la necesidad actual de sistemas capaces de entender y procesar descripciones humanas de manera efectiva, especialmente en escenarios donde las imágenes de consulta no están disponibles o no son suficientemente representativas.

Para alcanzar el objetivo general, se han establecido los siguientes objetivos específicos:

\begin{itemize}
    \item \textbf{O1. Revisión del estado del arte}: 
    Investigar las técnicas y modelos actuales de generación de imágenes a partir de texto, centrándose en las arquitecturas GAN, cGAN, AttnGAN y modelos de difusión como Stable Diffusion. Se evaluarán los enfoques existentes en la literatura para identificar las ventajas y limitaciones de cada metodología, así como las tendencias recientes en la combinación de visión por computador y procesamiento del lenguaje natural.
    
    \item \textbf{O2. Desarrollo de algoritmos de generación}: 
    Diseñar e implementar algoritmos que permitan transformar descripciones textuales en representaciones visuales coherentes y relevantes. Estos algoritmos serán evaluados utilizando diferentes conjuntos de datos con el objetivo de analizar su capacidad para manejar distintos niveles de complejidad semántica. Se prestará especial atención a la optimización del entrenamiento, la mejora de la calidad de las imágenes generadas y la reducción de la inestabilidad típica en redes generativas.
    
    \item \textbf{O3. Implementación de un prototipo funcional}: 
    Integrar los algoritmos desarrollados en la plataforma Java Multimedia Retrieval (JMR), habilitando un nuevo modo de consulta que permita a los usuarios introducir descripciones textuales en lenguaje natural y recuperar información visual relevante. Este prototipo combinará los resultados de la generación de imágenes con los mecanismos clásicos de búsqueda CBIR, ofreciendo así un sistema híbrido que optimiza tanto la accesibilidad como la precisión de las búsquedas.
\end{itemize}

\vspace{0.5cm}

El cumplimiento de los objetivos planteados tendrá un impacto significativo tanto a nivel práctico como académico:

\begin{itemize}
    \item \textbf{Aplicaciones prácticas en la industria y la investigación}: La posibilidad de realizar búsquedas visuales a partir de texto abre nuevas oportunidades en campos como la educación (búsqueda de materiales visuales educativos específicos), la medicina (búsqueda de imágenes médicas por descripción de patologías), la seguridad (generación de imágenes de referencia en investigaciones) y el comercio electrónico (búsqueda de productos basados en descripciones de características).
    
    \item \textbf{Avance en la integración de IA generativa y recuperación de información}: Este proyecto se sitúa en el cruce entre la visión por computador, el procesamiento de lenguaje natural y los sistemas de recuperación de información, contribuyendo al conocimiento actual mediante el estudio y evaluación de arquitecturas generativas para tareas de recuperación de imágenes.
    
    \item \textbf{Facilitación de interacciones más naturales con sistemas de información}: Al permitir consultas basadas en lenguaje natural, se mejora considerablemente la experiencia de usuario, haciendo más accesibles los sistemas de recuperación visual para personas no expertas en tecnologías digitales.
    
    \item \textbf{Apertura a futuras líneas de investigación}: El prototipo desarrollado servirá como base para futuras investigaciones, como la generación de vídeos a partir de texto, la mejora de la precisión semántica en modelos generativos o la combinación de información multimodal (texto, audio, imagen) en sistemas CBIR.
\end{itemize}

En definitiva, este proyecto se enmarca en una tendencia creciente hacia sistemas más inteligentes, adaptativos y centrados en el usuario, donde la comprensión semántica del lenguaje natural y su traducción a contenido visual de alta calidad se convierten en un componente clave para la innovación tecnológica.
